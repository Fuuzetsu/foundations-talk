\documentclass{beamer}
\usetheme{Warsaw}
\title[Dependent Types]{Quick look at Dependent Typing}
\author{Mateusz Kowalczyk}
\institute{University of Bath}
\begin{document}
\begin{frame}
  \titlepage
\end{frame}
\begin{frame}
  \frametitle{Aims}
  Very quickly, the aim of this presentation is to
  \pause
  \begin{itemize}[<+->]
    \item blast through basic concepts underlying type theory
    \item touch upon what dependent typing is
    \item spark interest in type theory
    \item encourage you to learn everything mention properly
    \end{itemize}
\end{frame}
\begin{frame}
  \frametitle{Aims}
  This presentation is not meant to
  \pause
  \begin{itemize}[<+->]
    \item present any new concepts
    \item provide comprehensive overview of any of the topics
    \item scare you
  \end{itemize}
  \pause
  There's a lot of stuff that only gets skimmed over.\\
  There's even more stuff that's outright missing.\\
  \pause
  \begin{itemize}
    \item I actually wrote up a fair amount of slides on untyped and
      typed $\lambda$-calculus but the presentation was too long
      before even getting into the topic of this talk!
  \end{itemize}
\end{frame}
\begin{frame}
  \frametitle{What is a type?}
  \framesubtitle{The Curry Howard Isomorphism}
  Common interpretation of Curry Howard isomorphism is to see types as
  propositions.
  \pause
  \begin{exampleblock}{Notation}
    We can read $p:P$ as `$p$ is a member of type $P$' or `$p$ is a proof of the
    proposition $P$'
  \end{exampleblock}
  \pause
  Ties in with intuitionistic (constructive) logic. Intuitionistic
  logic differs from classical logic in that:
  \pause
  \begin{itemize}[<+->]
    \item a statement is only true if there is a constructive proof of
      it
    \begin{itemize}
      \item a constructive proof requires us to provide a witness
          (an example) to make for a valid proof
    \end{itemize}
    \item no longer the question of `either true or false'
  \end{itemize}
  \pause
  $p$ is our witness of $P$.
\end{frame}

\end{document}
%%% Local Variables:
%%% mode: latex
%%% TeX-master: t
%%% End:
