\documentclass{beamer}
\usetheme{Warsaw}
\title[Dependent Types]{Quick look at Dependent Typing}
\author{Mateusz Kowalczyk}
\institute{University of Bath}
\begin{document}
\begin{frame}
  \titlepage
\end{frame}
\begin{frame}
  \frametitle{Aims}
  Very quickly, the aim of this presentation is to
  \pause
  \begin{itemize}[<+->]
    \item blast through basic concepts underlying type theory
    \item touch upon what dependent typing is
    \item spark interest in type theory
    \end{itemize}
\end{frame}
\begin{frame}
  \frametitle{Aims}
  This presentation is not meant to
  \pause
  \begin{itemize}[<+->]
    \item present any new concepts
    \item provide comprehensive overview of any of the topics
    \item scare you
  \end{itemize}
  \pause
  There's a lot of stuff that only gets skimmed over!
\end{frame}
\begin{frame}
  \frametitle{What is a type?}
  \framesubtitle{The Curry Howard Isomorphism}
  Common interpretation of Curry Howard isomorphism is to see types as
  propositions.
  \pause
  \begin{exampleblock}{Notation}
    We can read $p:P$ as `$p$ is of type $P$' or `$p$ is a proof of the
    proposition $P$'
  \end{exampleblock}
  \pause
  Ties in with intuitionistic (constructive) logic. Intuitionistic
  logic differs from classical logic in that:
  \pause
  \begin{itemize}[<+->]
    \item a statement is only true if there is a constructive proof of
      it
    \begin{itemize}
      \item a constructive proof requires us to provide a witness
          (an example) to make for a valid proof
    \end{itemize}
    \item no longer the question of `either true or false'
  \end{itemize}
  \pause
  $p$ is our witness of $P$.
\end{frame}
\begin{frame}
  \frametitle{Underlying concepts}
  \framesubtitle{Untyped $\lambda$-calculus }
  You should have at least heard of $\lambda$-calculus by now.
  \pause
  \begin{itemize}[<+->]
    \item Devised by Alonzo Church to investigate functions, function
      application and recursion.
    \item Most influential on functional programming languages (such as ML,
      Haskell and Miranda).
  \end{itemize}
\end{frame}
\begin{frame}
  \frametitle{Underlying concepts}
  \framesubtitle{Untyped $\lambda$-calculus }
  Some basic ideas
  \pause
  \begin{itemize}[<+->]
    \item individual variables
      \begin{itemize}
        \item $v_0$, $v_1$, $x$, $y$
      \end{itemize}
    \item abstraction
      \begin{itemize}
        \item $(\lambda x.e)$ is a function that returns $e$, given a
          formal parameter $x$
      \end{itemize}
    \item application
      \begin{itemize}
        \item $(e_1e_2)$ is an application of expression $e_1$ to
          $e_2$
        \item application is done by substitution: applying $(\lambda f.\lambda
          x.f\:x)$ to $g$ gives us $(\lambda x.g\:x)$ where $g$ is
          where $f$ was substituted by $g$
        \item in general, $(\lambda x.e_1)e_2$ is just $e_1[e_2/x]$
      \end{itemize}
  \end{itemize}
  \pause
  Functions can (and do) take and return functions as arguments!
\end{frame}
\begin{frame}
  \frametitle{Underlying concepts}
  \framesubtitle{Untyped $\lambda$-calculus}
  All functions have arity of 1. If we had $+$ defined in our language,
  $(\lambda x.\lambda y. x + y)$ is a function that takes $x$ and
  returns a function that takes $y$ and returns the result of $+$ on
  $x$ and $y$.\\
  \pause
  For convenience, I'll refer to functions as taking more arguments.
  \pause
  \begin{exampleblock}{Identity function}
    $(\lambda x.x)$ is just a function that returns whatever it's
    given.
  \end{exampleblock}
  \pause
  \begin{exampleblock}{Composition function}
    $(\lambda f.\lambda g.\lambda x.f\:(g\:x))$ is the mathematical
    composition function: think ``f(g(x))'' in other languages.
  \end{exampleblock}
  \pause
  \begin{exampleblock}{A `flip' function}
    $(\lambda f.\lambda x.\lambda y.f\:y\:x)$ is a function
    that takes a function and 2 other parameters and applies $f$ to
    $y$ and then applies the result to $x$. $f\:y\:x$ is the same as $(f\:y)\:x$.
  \end{exampleblock}
\end{frame}
\end{document}
%%% Local Variables:
%%% mode: latex
%%% TeX-master: t
%%% End:
