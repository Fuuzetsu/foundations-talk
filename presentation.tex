\documentclass{beamer}
\usepackage{listings, amssymb}
\usetheme{Warsaw}
\title[Monoid explained]{Monoid explained to an imperative programmer}
\author{Mateusz Kowalczyk}
\institute{University of Bath}
\lstset{language=Java}
\begin{document}
\begin{frame}
  \titlepage
\end{frame}
\begin{frame}
  \frametitle{Quick overview}
  Originally this talk was going to be about dependent types.\\
  \pause
  It turned out that I was running out of time before even getting
  into regular typing.\\
  \pause
  Changed the topic to something that builds upon knowledge you most
  likely already have: imperative programming.\\
  \pause
  This talk is going to essentially be a cut-down, partial, 10-minute
  version of the excellent presentation given by Brian Beckman titled
  ``Don't fear the Monad''.\\
  \pause
  Go watch it, it's awesome.
\end{frame}
\begin{frame}
  \frametitle{Quick overview}
  \framesubtitle{Presentation structure}
  Structure of this presentation
  \begin{itemize}[<+->]
    \item Functions
      \begin{itemize}
        \item more friendly notation and function composition
      \end{itemize}
    \item Generic monoid definition
    \item Simple monoid example (set of integers under addition)
    \item List monoid example
    \item Why would we ever want to abstract to a monoid?
  \end{itemize}
\end{frame}
\begin{frame}
  \frametitle{Functions}
  \framesubtitle{Notation}
  Here I'm going to define a notation that's more common in functional
  programming languages. The imperative language snippets are going to
  be in Java.\\
  Here we have a simple method from $int$ to $int$
  \pause
  \lstinputlisting{snippets/SingleVar.java}
  \pause
  We see that $x$ is given type $int$.\\
  From here on, we shall write this down as $x:int$.\\
  \pause
  In general, for all types $\tau$, $v:\tau$ is `v is a member of type $\tau$'.
\end{frame}
\begin{frame}
  \frametitle{Functions}
  \framesubtitle{Notation}
  Time for the same thing but on a function. I ask you to look past
  the fact that Java only has methods; making those methods static
  aims to lessen the difference.
  \pause
  \lstinputlisting{snippets/BasicFunc.java}
  \pause
  Here we have a very simple function $f$ that takes an $int$ (and binds it to
  $x$) and returns an $int$.\\
  \pause
  With the new notation, we can write the type of $f$ as $f:int
  \rightarrow int$.\\
  \pause
  Note how $x:int$ and $f:int \rightarrow int$ use the same notation.
  \pause
  We also define $a \rightarrow b \rightarrow c \equiv (a
  \rightarrow b) \rightarrow c$, i.e. the type signature is left associative.
\end{frame}
\begin{frame}
  \frametitle{Functions}
  \framesubtitle{Notation}
  Time for simple function application. Consider the same code again.
  \pause
  \lstinputlisting{snippets/BasicFunc.java}
  \pause
  We know that $f:int \rightarrow int$. Now consider any $v:int$. In
  Java, we might use the function like \lstinline|a = f(v);|. What is
  the type of $a$? Well, we take a $f:int \rightarrow int$ and feed an
  $int$ to it so we get an $int$ back.\\
  \pause
  With our new notation, we shall write \lstinline|f(v);| as $f\:v$.\\

\end{frame}
\begin{frame}
  \frametitle{Functions}
  \framesubtitle{Composition}
  Consider the following.
  \lstinputlisting{snippets/TwoBasicFunc.java}
  \pause
  We have $f:int \rightarrow int$ and $g:int \rightarrow int$.\\
  Let's say we want to apply $g$ first and then $f$ first. In Java
  we'd right \lstinline|f(g(v));|. With our new notation we write
  $f\:g\:v$. Note that application is left associative here.\\
\end{frame}
\begin{frame}
  \frametitle{Functions}
  \framesubtitle{Composition}
  Let's not stop on $int$s though and go generic.
  \lstinputlisting{snippets/Identity.java}
  \pause
  Here we have a function $id:A \rightarrow A$; it's a trivial
  function that just returns its argument. What's important is that
  it accepts any type: \lstinline|<A>| means a generic type $A$ in
  Java.\\
  \pause
  In our notation, this is simply $id:A \rightarrow A$: nothing
  special about it!
\end{frame}
\begin{frame}
  \frametitle{Functions}
  \framesubtitle{Composition}
  We can mix types without a problem! Consider $f:b \rightarrow c$, $g:a
  \rightarrow b$ and $x:a$.\\
  \pause
  We can very easily see that $g\:x:b$. So what's the type of
  $f\:(g\:x)$? Well, it's just $c$! $g\:x$ provides us a value of type
  $b$ and we apply $f:b \rightarrow c$ to it.\\
  \pause
  Mind that we need the parenthesis in $f\:(g\:x)$ as we have defined
  function application to be left associative.\\
  Without parenthesis, we'd get $f\:g\:x \equiv (f\:g)\:x$. But we
  can't have $f\:g$ because $f$ takes a value of type $b$ and $g$ is
  of type $a \rightarrow b$.\\
\end{frame}
\begin{frame}
  \frametitle{Functions}
  \framesubtitle{Composition}
  Before we proceed, I'd like to draw attention to the fact that the
  type notation for `variables' and for functions is basically the
  same.\\
  \pause
  In fact, we can think of $x:a$ as a nullary function of type $a$: a function that takes
  no arguments! Consider a function with the following signature then:
  $applier:(a \rightarrow b) \rightarrow a \rightarrow b$. It might
  look different than all the previous ones but let's try to read
  it.\\
  \pause
  $(a \rightarrow b)$ is just a function that takes a value of type
  $a$ and returns a value of type $b$. So what's $(a \rightarrow b)
  \rightarrow a \rightarrow b$? Well, it's just a function that takes a value of
  type $a \rightarrow b$, a value of type $a$ and returns a $b$. The
  first argument just happens to be a function!\\
  \pause
\end{frame}
\begin{frame}
  \frametitle{Functions}
  \framesubtitle{Composition}
  Continuing from the previous slide, I can show that making such a
  function is fairly trivial. We need to define syntax for creating
  functions first though.\\
  \pause
  \lstinputlisting{snippets/BasicFunc.java}
  We can write this function like this:
  \lstinputlisting{snippets/HaskFunc}
  \pause
\end{frame}
\begin{frame}
  \frametitle{Functions}
  \framesubtitle{Composition}
  So going back to our $(a \rightarrow b)
  \rightarrow a \rightarrow b$ function, how can we make one? Let's
  call this function $h$:
  \lstinputlisting{snippets/Applier}
  \pause
  Simple stuff! It turns out that $h$ is just function application
  that we've been doing all the time!
\end{frame}
\begin{frame}
  \frametitle{Functions}
  \framesubtitle{Composition}
  Isn't $h\:f\:x = f\:x$ pointless though? Why not just write $f\:x$?\\
  \pause
  The idea is that we just wrote a function $h$ that takes any
  function of type $a \rightarrow b$ and a value of type $a$. As we saw with $id$,
  $a$ and $b$ can be anything: the types are polymorphic.
\end{frame}
\begin{frame}
  \frametitle{Functions}
  \framesubtitle{Composition}
  When we previously did a $f\:(g\:x)$, what we really wanted is a
  function that does $g$ and then $f$ on $x$, in order.\\
  Here $f:b \rightarrow c$, $g:a \rightarrow b$ and $x:a$.
  \pause
  So we want some $h:a \rightarrow c$. How can we produce it? This is
  where function composition steps in.\\
  \pause
  What we need is a function $h$ with a signature $(b \rightarrow c)
  \rightarrow (a \rightarrow b) \rightarrow (a \rightarrow c)$.\\
  So: $h\:f\:g\:x = f\:(g\:x)$.\\
  \pause
  That's it! It's that simple. This operation is so common that we
  define $h\:f\:g\:x = f\:(g\:x)$ to be $(f \circ g)\:x$. So just $(f
  \circ g)$ is of type $a \rightarrow c$ and $(f \circ g)\:x$ is of
  type $c$.\\
  \pause
  We can give this new function $(f \circ g)$ a name: $f' = f \circ g$.
\end{frame}
\begin{frame}
  \frametitle{Monoids}
  \framesubtitle{Monoid definition}
  Now that we have a sane notation and function composition, we can
  finally define a monoid. A monoid is a set $S$ along with a binary
  operation $\cdot$ satisfying three simple laws:
  \pause
  \begin{itemize}[<+->]
    \item Closure
      \begin{itemize}
        \item $\forall x,y \in S: x \cdot y \in S$
      \end{itemize}
    \item Associativity
      \begin{itemize}
        \item $\forall x,y,z \in S: (x \cdot y) \cdot z = x \cdot (y
          \cdot z)$
      \end{itemize}
    \item Identity
      \begin{itemize}
        \item $\exists e \in S: \forall x \in S: e \cdot x = x \cdot e
          = x$
        \item Monoids where $x \cdot y = y \cdot x$ are called
          commutative monoids
      \end{itemize}
  \end{itemize}
\end{frame}
\begin{frame}
  \frametitle{Monoids}
  \framesubtitle{$\mathbb{Z}$ under addition}
  We can easily show that a set of all integers ($\mathbb{Z}$) is a monoid
  under addition: that is, we use $+$ for our $\cdot$.
  \pause
  \begin{itemize}[<+->]
    \item Closure
      \begin{itemize}
        \item $\forall x,y \in \mathbb{Z} : x + y \in \mathbb{Z}$
      \end{itemize}
    \item Associativity
      \begin{itemize}
        \item $\forall x,y,z \in \mathbb{Z} : (x + y) + z = x + (y + z)$
      \end{itemize}
    \item Identity
      \begin{itemize}
        \item $\forall x \in \mathbb{Z} : 0 + x = x + 0 = x$
        \item Here $0$ is our identity element
        \item Happens to be a commutative monoid thanks to addition
      \end{itemize}
  \end{itemize}
  \pause
  We can also very easily show that $\mathbb{Z} / \{0\}$ under multiplication is
  also a monoid: we use $1$ as the identity element.
\end{frame}
\begin{frame}
  \frametitle{Monoids}
  \framesubtitle{Functions under composition}
  Having talked about functions and composition, it's time to put the
  to use. Consider a set of all functions of type $\tau \rightarrow \tau$,
  F. We can form a monoid using $\circ$ as $\cdot$.
  \pause
  \begin{itemize}[<+->]
    \item Closure
      \begin{itemize}
        \item $\forall f,g \in F : f \circ g \in F$
        \item $((f:\tau \rightarrow \tau) \circ (g:\tau \rightarrow \tau)):\tau \rightarrow \tau$
      \end{itemize}
    \item Associativity
      \begin{itemize}
        \item $\forall f,g,h \in F : (f \circ g) \circ h = f
          \circ (g \circ h)$
        \item $\forall (x:\tau),\forall f,g,h \in F : (f \circ g) \circ
          h\:x = f\:(g \circ h)\:x = f\:(g\:(h\:x))$
        \item $\forall (x:\tau),\forall f,g,h \in F : f \circ (g \circ
          h)\:x = f\:((g \circ h)\:x) = f\:(g\:(h\:x))$
      \end{itemize}
    \item Identity
      \begin{itemize}
        \item $\forall f \in F: id \circ f = f \circ id = f$
        \item Here $id\:x = x$ is our identity element. $id$ is
          polymorphic so it gets the type $(\tau \rightarrow \tau)
          \rightarrow (\tau \rightarrow \tau)$
        \item $\forall (x:\tau),\forall f \in F: (f \circ id)\:x = f\:(id\:x) = f\:x$
        \item $\forall (x:\tau),\forall f \in F: (id \circ f)\:x =
          id(f\:x) = f\:x$
        \item Mind that by definition of $id$, $id\:f\:x = f\:x$ and
          $f\:(id\:x) = f\:x$. Not applying to $x$, we just get $f$ in
          both cases.
      \end{itemize}
  \end{itemize}
\end{frame}
\end{document}
%%% Local Variables:
%%% mode: latex
%%% TeX-master: t
%%% End:
